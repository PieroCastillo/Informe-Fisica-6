\documentclass[../main.tex]{subfiles}

\begin{document}
En conclusión, este informe experimental ha permitido estudiar con precisión el comportamiento de un sistema de resorte oscilante y analizar los resultados obtenidos.
Se ha demostrado que el ajuste de la recta en la gráfica de los datos experimentales es altamente preciso, respaldado por un coeficiente de determinación ($R^2$) cercano a la unidad, lo que indica una excelente concordancia con el modelo teórico.

Se ha determinado la constante del resorte ($k$) con una alta exactitud, validando la metodología empleada y respaldando la confiabilidad de los datos experimentales.
La evaluación de la incertidumbre asociada a esta constante ha revelado un error relativo menor al \qty{1}{\percent}, lo cual confirma la alta precisión de la medida y brinda una base sólida para futuros cálculos y análisis.

Asimismo, se ha verificado que el movimiento estudiado se asemeja de manera significativa a un Movimiento Armónico Simple (MAS) mediante el cálculo de la frecuencia de manera teórica y comparándola con los experimentales procesados en Excel, cumpliendo de esta manera, las condiciones fundamentales de una fuerza de restitución proporcional al desplazamiento y una aceleración opuesta y proporcional al desplazamiento.
Estos hallazgos respaldan la validez del modelo teórico y confirman la naturaleza periódica y oscilatoria del sistema estudiado.

La gráfica del periodo al cuadrado en función de la masa ha demostrado una relación lineal consistente con la ley teórica, donde la pendiente representa la constante del resorte ($k$).
El cálculo de la incertidumbre asociada a esta pendiente ha mostrado una precisión aceptable, con un error relativo inferior al \qty{1.1}{\percent}, lo cual valida la confiabilidad de los resultados obtenidos y su coherencia con las expectativas teóricas.

En resumen, este estudio experimental ha proporcionado una sólida evidencia científica de que el sistema de resorte oscilante se aproxima de manera precisa y confiable a un Movimiento Armónico Simple.
Los resultados obtenidos y su alta precisión respaldan la validez de las mediciones experimentales realizadas y sientan las bases para una comprensión más profunda y aplicaciones futuras de los principios del Movimiento Armónico Simple en diversos contextos científicos y tecnológicos.
\end{document}
