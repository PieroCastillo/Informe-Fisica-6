\documentclass[../main.tex]{subfiles}

\begin{document}

\subsection{Movimiento Periódico}

Un movimiento periódico se repite
en un ciclo definido; se presenta siempre que un cuerpo
tiene una posición de equilibrio estable y una fuerza de
restitución que actúa cuando el cuerpo se desplaza a partir
del equilibrio.  \parencite{sears}

\subsection{Movimiento Armónico Simple}

Si en el movimiento periódico
la fuerza de restitución $\vec{F}_x$ es directamente proporcional al
desplazamiento x, el \textit{movimiento se denomina armónico
simple (MAS)}. \parencite{sears}

\subsection{Ecuación del Movimiento Armónico Simple}

Según la condición del movimiento armónico simple:
\begin{equation} \label{force_res}
    \vec{F}_x = - k\cdot \vec{x}
\end{equation}

y usando la Segunda Ley de Newton:
\begin{equation} \label{second_newton_law}
    \vec{F}_r = m\cdot\vec{a}_r
\end{equation} 

Sabiendo que la fuerza restauradora $\vec{F}_x$ es la única fuerza que actua en el cuerpo,
procedemos a reemplazar \ref{force_res} en \ref{second_newton_law}:

\begin{equation} \nonumber
    \vec{F}_x = m\cdot\vec{a}
\end{equation}
\begin{equation} \nonumber
    - k\cdot \vec{x} = m\cdot\vec{a}
\end{equation}
Ordenando:
\begin{equation} \label{mas_eq}
    m\cdot\frac{d^2\vec{x}}{dt^2} + k\cdot \vec{x} = 0
\end{equation}
La ecuación \ref{mas_eq}, debido a que posee una función con sus derivadas, es una \textit{ecuación
diferencial}; esta ecuación describe la posición de una partícula que desarrolla un MAS.\\ \\
La solución de la ecuación \ref{mas_eq} es la siguiente: 
\begin{equation} \label{mas_general_sol}
    x = Acos(\omega t + \phi_0)
\end{equation}

Donde:

\begin{itemize}
    \item $\omega$: frecuencia angular
    \item $\phi_0$: fase inicial
    \item $A$: amplitud
    \item $x$: posición de la partícula
\end{itemize}

\subsection{Frecuencia}
Es el número de ciclos ejecutados por cada segundo. \cite{sears}
\begin{equation} \label{freq}
    f\:=\:\frac{n\acute{u}mero\:de\:ciclos}{\Delta t}
\end{equation}

\subsection{Frecuencia Angular}

Se define como $2\pi$ veces el número de ciclos ejecutados cada segundo.

\begin{equation} \label{freq_ang_1}
    \omega\:=\:2\pi \cdot f
\end{equation}

También es posible relacionar la frecuencia angular con la constante \textit{k} y
con la masa de la partícula por medio de la siguiente ecuación:

\begin{equation} \label{freq_ang_2}
    \omega\:=\: \sqrt{\frac{k}{m}}
\end{equation}

\subsection{Amplitud}

Es la elongación máxima.\cite{web_mas}

\subsection{Fase Inicial}

Se trata del ángulo que representa el estado inicial de vibración, 
es decir, la elongación x del cuerpo en el instante t = 0. \parencite{web_mas}

\end{document}